%! TEX root = ../000-main.tex
\chapter{Generalized non-parametric regression model}
\chaptermark{G NP reg. model}

\section{Introduction}
Non-parametric version of the Generalized Linear Model (GLM).

\begin{definition}{Generalized Linear Model (GLM)}{GLM}
A GLM is a model of the form
\begin{equation}
y = f(\eta) + \epsilon
\end{equation}
where $y$ is a random variable, $\eta$ is a linear predictor, $f$ is a \iemph{link function},
and $\epsilon$ is a random error term.
\end{definition}

\pagebreak
\section{Binary Response}

\subsection{Logistic Regression}
First, we consider the standard logistic regression model.

\begin{definition}{Logistic Regression}{logistic}
    A GLM model with a binary response.

    Let $(X, Y)$ be two r.v., where $X$ is continuous and $Y$ is binary. We
    assume that $(Y \mid X = x) \sim \text{Bernoulli}(p(x))$ and that
    there exist parameters $\beta_0, \beta_1$ such that:
    \begin{equation*}
        \underbrace{\log\left(\frac{p(x)}{1 - p(x)}\right)}_{\text{Logit}(p(x))}
        = \beta_0 + \beta_1 x \iff
        \underbrace{\frac{e^{\beta_0 + \beta_1 x}}{1 + e^{\beta_0 + \beta_1 x}} = p(x)}_{\text{Logistic} = p(x)}
    \end{equation*}
    \tcblower
    In logistic regression, the link function is the \iemph{logit function}, because
    if links the conditional expectation $P(x) = \mathds{E}(Y \mid X = x)$
    with the linear predictor $\beta_0 + \beta_1 x$.
\end{definition}

\subsubsection{Estimation of parameters}
Let $(x_i, y_i)$ be a sample of size $n$ from the population $(X, Y)$ following
the logistic regression model. The estimation of the parameters $\beta_0, \beta_1$
is done by maximization of the log-likelihood function:
\begin{equation*}
    \ell(\beta_0, \beta_1) = \sum_{i=1}^n \left(y_i \log \left( \frac{p(x_i)}{1 - p(x_i)} \right) + \log\bigl(1 - p(x_i)\bigr) \right)
\end{equation*}

This maximization is done by numerical methods. The most used algorithm
(equivalent to the \iemph{Newton-Raphson} method)
is known as \iemph{iteratively re-weighted least squares} (IRWLS), and
is also used for fitting other GLM models.

\begin{definition}{Iteratively re-weighted least squares (IRWLS)}{IRWLS}\index{IRWLS}
    The IRWLS algorithm is an iterative algorithm for fitting GLM models.
    It is based on the following steps:
    \begin{enumerate}
        \item Start with an initial guess for the parameters $\beta_0, \beta_1$.
        \item Compute the conditional expectation $P(x_i) = \mathds{E}(Y \mid X = x_i)$
            for each $i$.
        \item Compute the weights $w_i = P(x_i) \cdot (1 - P(x_i))$.
        \item Compute the weighted least squares estimator of the parameters
            $\beta_0, \beta_1$.
        \item Repeat steps 2 to 4 until convergence.
    \end{enumerate}
\end{definition}

\subsection{Non-parametric Binary Regression}
We now consider the non-parametric version of the logistic regression model.

\begin{definition}{Non-parametric binary regression}{npbinereg} 
    Non-parametric version of the logistic regression model (def~\ref{def:logistic}).

    The bivariate r.v. $(X, Y)$ has a joint distribution such that
    $(Y \mid X = x) \sim \text{Bernoulli}(p(x))$, with:
    \begin{equation*}
        \log\left(
            \frac{p(x)}{1 - p(x)} = \theta(x)
        \right)
    \end{equation*}
    where $\theta(x)$ is a smooth function of $x$.

    Equivalently,
    \begin{equation*}
        p(x) = \frac{e^{\theta(x)}}{1 + e^{\theta(x)}}
    \end{equation*}

    \tcblower

    The logit link ($\log\left(\frac{p}{1 - p}\right)$) is also used in this context.

    The function $\theta(x)$ is \iemph{free of constraints}, while $p(x) \in [0, 1]$.

    $p(x) = \mathds{E}(Y \mid X = x)$ is the regression function.
\end{definition}


\pagebreak
\section{Generalized non-parametric regression model}
\sectionmark{G NP reg. model}
Let's now consider the general case, for any response variable.

\begin{definition}{Generalized non-parametric regression model}{}

    The bivariate r.v. $(X, Y)$ has a joint distribution such that:
    \begin{equation*}
        (Y \mid X = x) \sim f(y;\; m(x),\, \psi)
    \end{equation*}
    where $m(x) = \mathds{E}(Y \mid X = x)$ is a smooth function of $x$,
    possibly subject to certain constraints (non-negativity, boundedness, etc.)
    and $\psi$ represents other parameters (e.g. variance) not depending on $x$.

    There exists an invertible link function $g$ such that:
    \begin{equation*}
        \theta(x) = g(m(x)), \quad m(x) = g^{-1}(\theta(x))
    \end{equation*}
    where $\theta(x)$ is a smooth function of $x$ free of constraints.

    \tcblower

    Alternatively, we can write:
    \begin{equation*}
        (Y \mid X = x) \sim f_2(y;\; \theta(x),\, \psi) = f(y;\; g^{-1}(\theta(x)),\, \psi)
    \end{equation*}

\end{definition}

\pagebreak
\section{Estimation by maximum local likelihood}

Let's focus on the non-parametric binary response model (def~\ref{def:npbinereg}).
We have $n$ data observed for this model, $(x_i, y_i)$, $i = 1, \ldots, n$,
and the goal is to estimate $p(t) = \mathds{E}(Y \mid X = t)$ for $t \in \mathbb{R}$
for a generic value $t \in \mathds{R}$.

Given that $\theta(x)$ is a smooth function, a first order
Taylor expansion of $\theta(x)$ around $t$ gives that for $x$ close to $t$:
\begin{equation*}
    \theta(x) \approx \theta(t) + \theta'(t) (x - t)
\end{equation*}

Then in a neighborhood of $t$, the standard logistic model is approximately
valid:
\begin{equation*}
    \theta(x) \approx \beta_0^t + \beta_1^t (x - t)
\end{equation*}

We can now fit this logistic model by \iemph{maximum local likelihood}.
The contribution to the log-likelihood function of each observation is:
\begin{equation*}
    y_i\log\left(
        \frac{p_i^t}{1 - p_i^t}
    \right) + \log\left(1 - p_i^t\right)
\end{equation*}
where
\begin{equation*}
    p_i^t = \frac{e^{\beta_0^t + \beta_1^t x_i }}{1 + e^{\beta_0^t + \beta_1^t x_i }}
    \approx p(x_i)
\end{equation*}

The closer $x_i$ is to $t$, the better is the approximation $p_i^t \approx p(x_i)$.

Adding up all the data contributions, weighted by the distance:
$w_i^t \propto K((t - x_i)/h)$, where $K$ is a kernel function and $h$ is a bandwidth,
we get the following log-likelihood function:
\begin{equation*}
    \ell(\beta_0^t, \beta_1^t) = \sum_{i = 1}^n w_i^t y_i\log\left(
        \frac{p_i^t}{1 - p_i^t}
    \right) + \log\left(1 - p_i^t\right)
\end{equation*}

\subsubsection{Maximizing the local log-likelihood}
We use a \iemph{weighted} version of the IRWLS algorithm (def~\ref{def:IRWLS}),
at each iteration, we multiply the usual weights $p_i^s(1 - p_i^s)$ by the
kernel weights $w_i^t$.

\pagebreak
\section{Bandwidth selection}
We present two possibilities (there are other alternatives):
\begin{itemize}
    \item Minimizing in $h$ the probability of miss-classification of a new observation
    \item Maximizing in $h$ the expected log-likelihood of a new observation
\end{itemize}
In both cases, the quantities must be estimated, which can be done by cross-validation.

\begin{definition}{Probability of miss-classification}{} of a new observation
    by LOOCV:

    This probability is estimated by:
    \begin{equation*}
        p_\text{CV}(h) = \frac{1}{n} \sum_{i = 1}^n \mathds{1}_{\{y_i \neq \hat{y}_i^{(-i)}\}}
    \end{equation*}
    where $\hat{y}_i^{(-i)} = \mathds{1}_{\{\hat{p}_i^{(-i)} \geq 1/2\}}$ and
    $p_i^{(-i)} = \hat p_h^{(-i)}(x_i)$ is the estimation of
    $p(x_i) = P(Y = 1 \mid X = x_i)$ when using $h$ as bandwidth and all the observations
    except the $i$-th one.
\end{definition}

\begin{definition}{Expected log-likelihood}{} of a new observation by LOOCV:

    The cross-validation estimation of the expected log-likelihood of
    a new observation when using $h$ as bandwidth is:
    \begin{align*}
        \ell_\text{CV}(h) &= \frac{1}{n} \sum_{i = 1}^n \log\left(
            \widehat{P}_h^{(-i)}(Y = y_i \mid X = x_i)
        \right) \\
                          &= \frac{1}{n} \sum_{i = 1}^n \log\left(
                              y_i \log\left(
                                  \frac{{p}_i^{(-i)}}{1 - {p}_i^{(-i)}}
                              \right) + \log\left(1 - {p}_i^{(-i)}\right)
                          \right)
    \end{align*}
    where ${p}_i^{(-i)} = \hat p_h^{(-i)}(x_i)$ is the estimation of
    $p(x_i) = P(Y = 1 \mid X = x_i)$ when using $h$ as bandwidth and all the observations
    except the $i$-th one.
\end{definition}
