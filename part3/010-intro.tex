%! TEX root = ../000-main.tex
\chapter{Introduction to Functional Data Analysis (FDA)}
\chaptermark{FDA Intro.}

\section{An overview of FDA}

Observing and saving complete functions as the result of random
experiments is possible by the development of real-time
measurement instruments and data storage resources.

\begin{example}{}{}
	For patients involved in a clinical trial, the blood pressure
	is monitored in continuous-time during 24 hours
\end{example}

\begin{definition}{Functional data}{}
	Samples where a whole function is observed at each sampling unit
	are referred to as \iemph{Functional Data}.
	\tcblower
	Random functions are \iemph{Statistical atoms} in this case.
\end{definition}

\begin{figure}[H]
	\begin{tikzpicture}[
			every node/.style={rectangle, draw,rounded corners,inner sep=10pt},
		]
		\node (u) {Univariate};
		\node[right=of u] (m) {Multivariate};
		\node[right=of m] (f) {Functional};
		\draw[->,thick] (u) -- (m);
		\draw[->,thick] (m) -- (f);
	\end{tikzpicture}
	\caption{Evolution of Statistical Analysis}
\end{figure}

Functional data analysis (FDA) deals with the statistical description
and modeling of samples of \iemph{random functions}.

Functional data can also be obtained from standard random samples,
by the application of non-parametric curve estimation methods:
\begin{itemize}
	\item Non-parametric density estimation.
	\item Non-parametric regression.
	\item Non-parametric version of the GLM.
\end{itemize}

Kernel estimators, local polynomial regression, local likelihood,
spline smoothing and interpolations \dots

Many standard statistical techniques have been extended to
functional data:
\begin{itemize}
	\item \textbf{Exploratory data analysis} (location and scale measures, \dots).
	\item \textbf{Regression models} (lm, glm, non-parametric, \dots).
	\item \textbf{Multivariate Analysis} (PCA, MDS, Clustering, Depth measures, \dots).
	\item \textbf{Hypothesis testing}.
	\item \textbf{Time Series}, \textbf{Spatial Statistics}, \dots
\end{itemize}

Others methods are specific for this kind of data because they
exploit the nature of functions:
\begin{itemize}
	\item \textbf{Principal differential analysis} is a kind of principal component
	      analysis made on the derivatives of the observed functions,
	\item \textbf{Registration} is a pre-process step where a change of variable is done
	      in each observed function in order to make them as similar as
	      possible.
\end{itemize}

\paragraph{Software}
In R, the \texttt{fda} package provides a set of tools for functional data analysis.
There is also \texttt{fda.usc} among others.

\sectionmark{Concepts of FA for FDA}%
\section{Concepts of Functional Analysis useful in FDA}

\begin{definition}{Spatial dependence}
	When dealing with spatial data, the \iemph{spatial dependence} is
	the correlation between the values of a variable at two different
	locations in space.
\end{definition}

\clearpage
\sectionmark{Formal definition}%
\section{Formal definition of functional data}%
\sectionmark{Formal definition}%

\begin{definition}{functional random variable}{frv}
	A random variable $\boldsymbol{\mathcal X}$ taking values in an infinite dimensional space (or
	functional space)
\end{definition}

\begin{definition}{functional data}{fd}
	An observation $\mathcal X$ of $\boldsymbol{\mathcal X}$ is called \iemph{functional data}.
	\tcblower

	Likewise a \iemph{functional data set} $\mathcal X_1,\ldots,\mathcal X_n$ is a collection of $n$
	observations of functional variables $\boldsymbol{\mathcal X_1}, \ldots, \boldsymbol{\mathcal X_n}$
	identically distributed as $\boldsymbol{\mathcal X}$.
\end{definition}

Let $T = [a,\,b] \subseteq \mathds R$, usually we work with functional data that are
elements of $L^2(T)$, the space of square integrable functions on $T$:
\begin{equation*}
	L^2(T) = \left\{ f : T \to \mathds R \,\middle|\, \int_a^b f^2(t) \,dt < \infty \right\}.
\end{equation*}

Smoothness of $\mathcal X$ as a function of $t \in T = [a,\, b]$ is implicitly assumed.

\begin{definition*}{Hilbert Space}{HS}
	A Hilbert space is a vector space endowed with an inner product.
\end{definition*}

\begin{definition}{Separable Hilbert Space}{}\index{Hilbert space}\index{separable Hilbert space}
	A separable Hilbert space is a Hilbert space that is separable as a metric space. That is,
	an infinite generalization of the usual Euclidean spaces in $\mathds R^p,\,p\in\mathds N^+$.
	\tcblower
	\begin{note}
		$L^2(T)$ is a separable Hilbert space with the inner product:
		\begin{equation*}
			\langle f, g \rangle = \int_T f(t) g(t) \,dt.
		\end{equation*}
	\end{note}
\end{definition}

\pagebreak
\subsection{Functional data and random processes}

\begin{definition}{Random process}{rp} (or \iemph{stochastic process})\index{random process}
	indexed by $T = [a,\,b] \subseteq \mathds R$
	is a random variable $\boldsymbol{\mathcal X}$ taking values in:
	\begin{equation*}
		L^2(T) = \left\{ f : T \to \mathds R \,\middle|\, \int_a^b f^2(t) \,dt < \infty \right\}.
	\end{equation*}
    \tcblower

    This is the same $L^2(T)$ as the one usually used for functional data.
\end{definition}

\begin{question*}{What is the difference between a functional data and a random process?}

	In random process analysis it is usually assumed that we can observe
    only \emph{one} realization of the process, while in functional data analysis
	it is always assumed that independent observations from the random
	process are available.
\end{question*}

% \section{Concepts of Functional Analysis useful in FDA}

