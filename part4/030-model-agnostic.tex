%! TEX root = ../000-main.tex
\chapter{Model-agnostic interpretability methods}

\section{Introduction}

Model-agnostic interpretation methods are those that only require
the evaluation of the fitted prediction model on the training set, on
the test set, or on perturbations of them.

No other information from the kind of model at hand is needed.

To be more formal, let $f$ be the prediction function estimated from a
training sample using a generic prediction model.

We assume that $f$ depends on $p$ arguments, so the predicted value
for $x = (x_1, \dots , x_p)$ is $f(x)$.

The only connection between the model-agnostic interpretation
methods and the prediction model is through the function $f$ and,
more specifically, only evaluations of $f$ at different points $x$ are
allowed.

Under this setting, to interpret the prediction model equals to
interpret the prediction function $f$.

This task is essentially the same we would have to do if we wanted
to explore a generic mathematical function $g$ depending on $p$
variables, from which only evaluations are allowed.

Therefore, any procedure that allows exploring a generic function $g$
(using only evaluations of $g$) can be considered a model-agnostic
method that could be used for interpreting a prediction function $f$.

For instance, computing a numerical approximation to the gradient
of $f$ at a point $x$ can be considered a model-agnostic interpretation
method, as well as using this approximation to compute the first
order Taylor expansion of $f$ around $x$.

Model-agnostic methods are specially useful for interpreting
prediction models for which there are no specific interpretation
methods.

Nevertheless, model-agnostic methods can also improve the
interpretation of models that are usually considered interpretable.

Even multiple linear regression could benefit from the application of
some model-agnostic methods.

When a new model-agnostic interpretation method is introduced, a
good practice is to check what it provides when applied to a classical
simple prediction model, as linear regression, logistic regression or
their additive extensions.

In these simple cases, sometimes it is possible to obtain the closed
expression of the new method results and then to relate them to the
standard outputs of the classical methods.

This way the new interpretation method will be either reinforced
(when its classical counterpart is a sensible measure) or called into
question (when the opposite happens).

\section{Global measures of variable relevance}

Let us consider the prediction problem involving the random
vector $(X,\,Z,\,Y),\;X\in\mathds{R}^p,\,Z\in\mathds{R},\,Y\in\mathds{R}$,
where $Y$ is the response variable that should be predicted from $(X,\,Z)$.

A \iemph{prediction function} $f : \mathds{R}^{p+1} \to \mathds{R}$ has
\emph{expected loss} (or \emph{risk}):
\begin{equation*}
    R(f(X,\,Z),\,Y) = \mathbb{E} \left[ L(f(X,\,Z),\,Y) \right]
\end{equation*}
where $L : \mathds{R}\times\mathds{R} \to \mathds{R}^+$ is a \iemph{loss function}
measuring the \emph{cost} associated with predicting $Y$ by $f(X,\,Z)$.

For instance, \iemph{quadratic loss} is defined as:
\begin{equation*}
    L(y,\,f(x,\,z)) = (y - f(x,\,z))^2
\end{equation*}

We consider the problem of measuring the effect of the single
variable $Z$ on the prediction function $f$ when predicting $Y$
by $f(X,\,Z)$. We call that effect the \iemph{variable relevance}
or \iemph{variable importance} of $Z$.

We assume that a training sample of size $n_1$ and
a test sample of size $n_2$ are available.

\subsection{Leave-one-covariate-out (LOCO)}

A simple approach to define the importance of the variable $Z$:
\begin{enumerate}
    \item Fit the model including both $X$ and $Z$.
    \item Fit the model including only $X$ (leave $Z$ out).
    \item The relevance of $Z$ by \iemph{LOCO} is given by the relative
        decrease in the prediction accuracy when $Z$ is omitted from the model.
\end{enumerate}

This approach is used, for instance, in multiple linear regression
to decide if a variable should be included in the model or not.

\begin{note}
    The model must be fitted \emph{twice}
\end{note}

\subsubsection{Relevance by LOCO at a populational level}

It could happen that there would exists a natural reduced version of
$f$, say $f_p$, depending only on $p$ variables such that $f_p(X)$ would be the prediction
of $Y$ when $Z$ is not available.

For instance, the natural reduced version of $f(X,\,Z) = \beta_0 + X^T\beta_x + Z\beta_z$
could be $f_p(X) = \beta'_0 + X^T\beta'_x$ for some $\beta'_0,\,\beta'_x$ possibly
different from $\beta_0,\,\beta_x$.


\subsection{Variable importance by random permutations}
\subsection{Relevance by ghost variables}
\subsection{Other relevance measures based on perturbations}
\subsection{The relevance matrix}
\subsection{Variable relevance measures as Shapley’s values}
\subsection{Global graphical methods}

% \subsection{Importance of variables through disturbances}
% \subsection{Importance based on the Shapley Value}
% \subsection{Partial dependency graph}
% \subsection{Cumulative local effects graphs}

\section{Local methods}

\subsection{LIME: Local interpretable model-agnostic explanations}
\subsection{Local importance based on the Shapley Value}
\subsection{SHAP: SHApley Additive ExPlanations}
\subsection{Break down graphics}
\subsection{ICE: Individual conditional expectation, or ceteris paribus chart}
