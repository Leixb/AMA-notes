%! TEX root = ../000-main.tex
\chapter{Interpretability methods for specific models}
\chaptermark{IML specific}

\begin{definition}{model-specific methods}{}
	Interpretability methods developed for a particular prediction method.

	They allow model exploration, validation or visualization.

	They require full access to the model structure.

	Different prediction models have different model-specific interpretability
	methods, usually difficult to be compared.

	\tcblower

	We are talking here about interpretability in random forests and neural
	networks. For support vector machines, refer to section 4.2.2 in
	Barredo-Arriet et al. (2020).
\end{definition}

\section{Random forests}

Random forests are combinations of more simpler models:
classification and regression trees (CART).

CART are usually considered transparent models because the
prediction rules they encode are easily understood by non-experts.

Additionally, a simple importance measure for the input variables
can be defined for CART: at each split in the tree, the
\emph{improvement in the split-criterion} is the importance measure
attributed to the splitting variable.

In random forests, this importance measure is accumulated over all trees in the forest
separately for each variable.

Breiman (2001) introduced an alternative way to measure the variable importance
in random forests, combining the use of the
\iemph{out-of-bag} (OOB) samples and the principle of
\iemph{randomly permuting} the values of each predictor in a test sample to
measure the decrease in accuracy.

\subsection{Brief review of CART and random forests}

Three-based methods divide the feature space into a set of regions,
and then fir a simple model (like a constant) in each region.

They are conceptually simple, yet powerful.

\begin{example}{}{}
	Consider a regression problem with continuous response $Y$ and
	inputs $X_1$ and $X_2$, each taking values in the unit interval.

	Let $\{R_1,\dots,R_5\}$ be a partition of the unit spare into 5 regions.

	The corresponding regression model predicts $Y$ with a constant
	$c_m$ in a region $R_m$ that is,
	\begin{equation*}
		\hat{f}(X_1,\,X_2) = \sum_{m=1}^5 c_m I_{R_m}(X_1\,,\,X_2)
	\end{equation*}

	\begin{figure}[H]
		\begin{tikzpicture}[
				scale=4.5,
			]
			% t1 = 0.4 t2=0.6  t4= 0.8 t3= 0.33
			% constants:
			\pgfmathsetmacro{\ta}{0.4}
			\pgfmathsetmacro{\tb}{0.4}
			\pgfmathsetmacro{\tc}{0.6}
			\pgfmathsetmacro{\td}{0.7}

			\draw (0,0) rectangle (\ta,\tb) node[pos=.5] {$R_1$};
			\draw (0,\tb) rectangle (\ta,1) node[pos=.5] {$R_2$};
			\draw (\ta,0) rectangle (\tc,1) node[pos=.5] {$R_3$};
			\draw (\tc,0) rectangle (1,\td) node[pos=.5] {$R_4$};
			\draw (\tc,\td) rectangle (1,1) node[pos=.5] {$R_5$};

			\node[anchor=north] at (\ta,0) {$t_1$};
			\node[anchor=east] at (0,\tb) {$t_2$};
			\node[anchor=north] at (\tc,0) {$t_3$};
			\node[anchor=west] at (1,\td) {$t_4$};

			\node[anchor=south,rotate=90] at (-0.15,0.5) {$X_2$};
			\node[anchor=north] at (0.5,-0.15) {$X_1$};

		\end{tikzpicture}
		\hfill
		\begin{tikzpicture}[
				scale=4.5,
			]
			\pgfmathsetmacro{\ta}{0.4}
			\pgfmathsetmacro{\tb}{0.4}
			\pgfmathsetmacro{\tc}{0.6}
			\pgfmathsetmacro{\td}{0.7}

			\draw (0,0) rectangle (1,1);
			\draw (0,0) rectangle (0.25,\td);
			\draw (0,0) rectangle (0.55,\td);

			\draw[fill=white] (0.25,0.4) rectangle (0.9,0.8);
			\draw[fill=white] (0,0.6) rectangle (0.8,0.9);

			\node[anchor=south,rotate=90] at (-0.1,0.5) {$X_2$};
			\node[anchor=north] at (0.5,-0.15) {$X_1$};

		\end{tikzpicture}
		\caption{Two examples of partitions of the unit square into 5 regions.}%
		\label{fig:cart}
	\end{figure}

	\Cref{fig:cart} we can see two examples of partitions, the left has been obtained
	by recursive binary partitioning and can be easily represented by a binary tree.

	% TODO: add a figure with binary tree

\end{example}

\subsubsection{Regression trees}

Our data set consists of $p$ inputs and a response, for each of $n$ observations:
$(\boldsymbol{x}_i,\,y_i),\, i=1,\dots,n$ with $\boldsymbol{x}_i=(x_{i1},\dots,x_{ip})^T$.

The algorithm needs to automatically decide on the splitting variables and split points.

Suppose first that we have a partition into $M$ regions $R_1,\dots,R_M$,
and we model the response as a constant $c_m$ in each region:
\begin{equation*}
	{f}(\boldsymbol x) = \sum_{m=1}^M c_m I_{R_m}(\boldsymbol x)
\end{equation*}

If we adopt as our criterion the minimization of the sum of squares,
$\sum_{i=1}^n (y_i - {f}(\boldsymbol x_i))^2$, then the optimal
value for $c_m$ is the average (ave) of $y_i$ in the region $R_m$:
\begin{equation*}
	\hat{c}_m = \text{ave}(y_i \mid \boldsymbol x_i \in R_m)
\end{equation*}

\paragraph{Finding the best binary partition} in terms of the minimization of the
sum of squares is generally \emph{computationally infeasible}.

Instead, a greedy strategy is adopted:
\begin{algorithm}{}{}
	Starting with all the data, consider a splitting variable $j$ and a split point $x$,
	and define the pair of half-planes:
	\begin{equation*}
		R_1(j,\,s) = \{ \boldsymbol x \in \mathds R^p \mid x_j \leq s \} \qquad
		R_2(j,\,s) = \{ \boldsymbol x \in \mathds R^p \mid x_j > s \}
	\end{equation*}
	The we seek the splitting variable $j$ and split point $s$ that solve:
	\begin{equation*}
		\min_{j,\,s} \left\{
		\min_{c_1} \sum_{\boldsymbol x_i \in R_1(j,\,s)} (y_i - c_1)^2
		\quad+\quad \min_{c_2} \sum_{\boldsymbol x_i \in R_2(j,\,s)} (y_i - c_2)^2
		\right\}
	\end{equation*}
	For any choice of $j$ and $s$, the inner minimization is solved by:
	\begin{equation*}
		\hat{c}_1 = \text{ave}(y_i \mid \boldsymbol x_i \in R_1(j,\,s)) \qquad
		\hat{c}_2 = \text{ave}(y_i \mid \boldsymbol x_i \in R_2(j,\,s))
	\end{equation*}
	For each splitting variable, the determination of the split point $s$ can
	be done very quickly, and hence by scanning through all of the inputs,
	determination of the best pair $(j,\,s)$ is feasible.

	Having found the best split, we partition the data into the two resulting regions and repeat the
	splitting process on each of the two regions. This process is repeated until a stopping
	criterion is met.
\end{algorithm}

\begin{definition}{Squared-error impurity measure}{}
	for the $m$-th region (or \emph{node}) $R_m$ with $N_m$ cases and average
	response $\hat{c}_m = \frac{1}{N_m} \sum_{i: \boldsymbol x_i \in R_m} y_i$,
	is defined as:
	\begin{equation*}
		Q_m(T) = \frac{1}{N_m} \sum_{i: \boldsymbol x_i \in R_m} (y_i - \hat{c}_m)^2
	\end{equation*}
\end{definition}

\subsubsection{Classification trees}

Assume now that the target is a classification outcome taking values $1,\dots,K$.

The only changes needed in the tree algorithm affect the criteria for
\emph{splitting nodes} and \emph{pruning the tree}.

In a node $m$, representing a region $R_m$ with $N_m$ observations, let
\begin{equation*}
	\hat{p}_{mk} = \frac{1}{N_m} \sum_{i: \boldsymbol x_i \in R_m} I(y_i = k)
\end{equation*}
the proportion of class $k$ observations in node $m$.

We classify the observations in node $m$ to class $k(m) = \arg\max_k \hat{p}_{mk}$,
the majority class in node $m$.

The squared-error node impurity measure $Q_m(T)$ used for regression is not
suitable for classification. Instead, we use one of the following alternatives:
\begin{align}
	\frac{1}{N_m} \sum_{i: \boldsymbol x_i \in R_m} I(y_i \neq k) & = 1 - \max_k \hat{p}_{mk} \tag{misclassification error}                     \\
	\sum_{k\neq k'} \hat{p}_{mk} \hat{p}_{m k'}                   & = \sum_{k=1}^K \hat{p}_{mk} (1 - \hat{p}_{mk}) \tag{Gini index}             \\
	                                                              & \sum_{k=1}^K \hat{p}_{mk} \log \hat{p}_{mk} \tag{cross-entropy or deviance}
\end{align}

\begin{note}
	Both the Gini index and the cross-entropy are lower for the splits producing pure nodes, that are probably
	preferable.

	For this reason, either the Gini index or cross-entropy should be used when growing the tree.

	To guide cost-complexity pruning, any of the three measures can be used but typically
	the misclassification rate is used.
\end{note}

\subsubsection{Instability of trees}

One major problem with trees is their high variance. Often, a small change in the data can result in a very
different series of splits, making interpretation somewhat precarious.

The major reason for this instability is the hierarchical nature of the process:
the effect of an error in the top split is propagated down to all of the splits below it.

Instability is the price to be paid for estimating a simple, tree-based structure of the
data.

We can use methods such as bagging and random forests to reduce the variance of the tree:

\index{Bagging}
\paragraph{Bagging} averages $B$ trees to reduce the variance.

% \paragraph{Random forests} are a variant of bagging that use a random subset of the
% variables at each split. In many problems, it outperforms bagging.

\subsubsection{Random forests}

In random forests, a large amount of random trees is generated and then they
are averaged.

It is hopped to reduce variance without increase in bias.

The first random ingredient: take a \emph{bootstrap sample}, choosing with
replacement $n$ random elements from the original data set:
\begin{equation*}
	\{
	(
	\boldsymbol x_i^*,\,y_i^*
	),\, i = 1,\dots,n
	\} \text{ randomly chosen from }
	\{
	(
	\boldsymbol x_i,\,y_i
	),\, i = 1,\dots,N
	\}
\end{equation*}
Several real data appear more than once in the bootstrap sample (around $2/3$ of them).
The rest (around $1/3$) do not belong to the bootstrap sample and we call them
\iemph{out-of-bag} (OOB) sample.

This idea is shared with bagging (\iemph{bootstrap aggregation}), but
random forest try to reduce the correlation between the trees,
without increasing the variance too much.

This is achieved in the tree-growing process through random selection of the
input variables.

\paragraph{Out-of-Bag samples}
An important feature of random forests is its use of out-of-bag (OOB) samples:
\begin{quote}
	For each observation $\boldsymbol z_i = (\boldsymbol x_i,\,y_i)$ in the
	training set, construct its random forest predictor by averaging only
	those trees corresponding to bootstrap samples in which $\boldsymbol z_i$
	was not included.
\end{quote}

An OOB error estimate is almost identical to that obtained by $n$-fold cross-validation.

Hence, unlike many other non-linear estimators, random forests can be fit in one sequence,
with cross-validation being performed along the way.

Once the OOB error stabilizes, the training can be stopped.

\subsection{Variable Importance based on node impurity measures}


Let $T$ be a tree found in the fitting process of a CART, and let
$|T|$ be the number of terminal nodes in $T$.

\begin{definition}{Total impurity measure}{} of $T$ is defined as:
	\begin{equation*}
		C(T) = \sum_{m=1}^{|T|} N_m Q_m
	\end{equation*}
\end{definition}

Assume that the $j$-th variable is used to split the node $R_r$ of $T$ into
two child-nodes, say $R_{r'}$ and $R_{r''}$, this way producing a new tree $T'$.

The \iemph{improvement in the impurity measure} (also known as
\iemph{improvement in the split-criterion}) is:
\begin{equation*}
	C(T) - C(T') = N_r Q_r - \left(N_{r'} Q_{r'} + N_{r''} Q_{r''} \right)
\end{equation*}

A simple \emph{importance measure} for the input variables can be defined for
\emph{CART}:
\begin{itemize}
	\item At each split in the tree, the improvement in the split-criterion
	      is attributed to the splitting variable as a partial measure of its
	      importance.
	\item The importance measure of a variable is the \emph{sum of
		      the partial measures} over all splits in which it is involved.
\end{itemize}

In \emph{Random Forests}, this importance measure is accumulated over \emph{all the trees in the forest}
separately for each variable:

At each split in each tree, the improvement in the split-criterion
is the importance measure attributed to the splitting variable, and
is accumulated over all the trees in the forest separately for each variable.

\subsection{Out-of-Bag Variable Importance}

% TODO: cite properly
Breiman (2010) introduced an alternative way to measure the \iemph{variable importance}
(or \iemph{prediction strength}) in random forests, combining the use
of the out-of-bag (OOB) samples as test samples, and the principle of
\emph{randomly permuting the values of each predictor in a test sample}
to measure the decrease in accuracy.

\begin{algorithm}{}{}
	When the $b$-th tree is grown, the OOB samples are passed down the tree, and the prediction
	accuracy is recorded:
	\begin{equation*}
		SSE_{oob}^b = \sum_{i \in oob_b} \left( y_i - T^b(\boldsymbol x_i) \right)^2
	\end{equation*}
	Then the values for the $j$-th variable are randomly permuted in the OOB samples,
	and the accuracy is again computed:
	\begin{equation*}
		SSE_{oob,\,\pi(j)}^b = \sum_{i \in oob_b} \left( y_i - T^b(\boldsymbol x_{i\{\pi(j)\}}) \right)^2
	\end{equation*}
	The decrease in accuracy as a result of this permuting is averaged over all trees, and is used
	as a measure of the importance of variable $j$ in the random forest:
	\begin{equation*}
		VI_{oob}(j) = \frac{1}{B}\sum_{b=1}^B \left( SSE_{oob,\,\pi(j)}^b - SSE_{oob}^b \right)
	\end{equation*}

	\tcblower

	The randomization effectively voids the effect of a variable, much like setting a coefficient
	to zero in a linear model.

	\begin{note}
		This does not measure the effect on the prediction if this variable
		were not available, because if the model was refitted without the variable,
		other variables could be used as surrogates.
	\end{note}
\end{algorithm}

\section{Neural networks}

\subsection{A brief review of neural networks}

\begin{definition}{Neural Networks}{} (NN) or \iemph{Artificial Neural Networks} (ANN)
	are a class of Machine Learning models inspired by the structure and function of biological
	neural networks.

	They try to mimic with mathematical functions the way biological neurons
	in the brain process information.

	\tcblower

	Here, we will deal only with \emph{one-hidden-layer} neural networks.
\end{definition}

\begin{definition}{One-hidden-layer neural networks}{}
	are non-linear parametric regression model represented by the following
	directed acyclic graph:

	\begin{center}
		\begin{neuralnetwork}[height=5]
			\tikzstyle{input neuron}=[neuron, fill=orange!70];
			\tikzstyle{output neuron}=[neuron, fill=blue!60!black, text=white];

			\inputlayer[count=5, bias=false, title=Input Layer]

			\hiddenlayer[count=3, bias=false, title=Hidden Layer]
			\linklayers

			\outputlayer[count=3, title=Output Layer]
			\linklayers
		\end{neuralnetwork}
	\end{center}

	At each node $\boldsymbol N$ the inputs are additively combined, then
	they are transformed by an \iemph{activation function} $\sigma$ and they
	result in:
	\begin{equation*}
		z_{\boldsymbol N} = \sigma \left( \sum_{\ell \in \text{ input of } \boldsymbol N} w_\ell x_\ell \right)
	\end{equation*}

\end{definition}

\begin{example}{}{} A neural network represented by the graph:
	\begin{center}
		\begin{neuralnetwork}[height=5]
			\tikzstyle{input neuron}=[neuron, fill=orange!70];
			\tikzstyle{output neuron}=[neuron, fill=blue!60!black, text=white];

			\newcommand{\xin}[2]{
				\ifthenelse{\numexpr#2<4\relax}{#2}{
					\ifthenelse{\numexpr#2=4\relax}{\dots}{p}
				}
			}
			\newcommand{\xhid}[2]{
				\ifthenelse{\numexpr#2<3\relax}{#2}{
					\ifthenelse{\numexpr#2=3\relax}{\dots}{m}
				}
			}
			\renewcommand{\xout}[2]{1}

			\inputlayer[count=5, bias=false, title=Input Layer, text=\xin]

			\hiddenlayer[count=4, bias=false, title=Hidden Layer, text=\xhid]
			\linklayers

			\outputlayer[count=1, title=Output Layer, text=\xout]
			\linklayers
		\end{neuralnetwork}
	\end{center}
	corresponds to the following function from $\mathds R^p \to \mathds R$:
	\begin{equation*}
		f(\boldsymbol x) = \sigma_2 \left(
		    w_0^{(2)} + \sum_{j=1}^p w_j^{(2)} \sigma_1 \left(
                w_{0j}^{(1)} + \sum_{\ell=1}^p w_{\ell j}^{(1)} x_\ell
            \right)
		\right)
	\end{equation*}
\end{example}

\subsection{Interpretability in Neural Networks}

A useful tool for interpretability in neural networks is to look at the
\emph{derivatives of the prediction function}:
\begin{equation*}
    f(\boldsymbol x) = 
    \beta_0 + \sum_{j=1}^m \beta_j \sigma \left( \alpha_{0j} + \sum_{\ell=1}^p \alpha_{\ell j} x_\ell \right)
\end{equation*}
with respect to each variable $x_\ell,\,\ell=1,\dots,p$:
\begin{equation*}
    \frac{\partial f}{\partial x_\ell} = \sum_{j=1}^m \beta_j \alpha_{\ell j} \sigma' \left( \alpha_{0j} + \sum_{\ell=1}^p \alpha_{\ell j} x_\ell \right)
\end{equation*}
The gradient of $f$ at $\boldsymbol x$ is often required:
\begin{equation*}
    \nabla f(\boldsymbol x) = \left( \frac{\partial f}{\partial x_1}, \dots, \frac{\partial f}{\partial x_p} \right)
\end{equation*}

These computations are easy to implement since partial derivatives can be computed
using \iemph{backpropagation}, even in NN with complex architectures.

\begin{definition}{Activation maximization}{} consists on searching
    for the input pattern (\iemph{prototype}) that produces a maximum
    model response for a quantity of interest
    (for instance, the estimated probability to belong to one of the classes when the response
    is qualitative).

    The found prototype indicates which characteristics in the data are mainly taken into account
    by the model.
\end{definition}

\subsubsection{Explanation}

Given a data point $\boldsymbol x \in \mathds R^p$, the objective is to \emph{explain}
why the NN produces the response prediction $f(\boldsymbol x)$ for it.

For explanation of NN decision, \iemph{sensitivity analysis} and
\iemph{simple Taylor decomposition} are considered in Montavon et al. (2018). % TODO: add reference

\begin{definition}{Sensitive analysis}{}
    the goal is to identify the input feature along which the largest local variation is produced around
    a given data point $\boldsymbol x$.

    A possibility is to compute the \iemph{relevance score} at $\boldsymbol x$
    for each feature $h$, that is, \emph{the square of the partial derivative with respect to
    the $h$-th variable} of the function codified by the NN.
\end{definition}

\begin{definition}{Simple Taylor decomposition}{}
    the NN function is approached at a given data point $\boldsymbol x$ by the \emph{first
    order Taylor expansion} which is then interpreted as any linear estimator, providing
    an explanation of how the NN function varies around $\boldsymbol x$.
\end{definition}

\begin{note}
    These procedures, as well as activation maximization, have limitations
    to show the general pattern of interactions among variables.
\end{note}

% TODO: convolutional NN references
