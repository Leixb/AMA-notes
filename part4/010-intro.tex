%! TEX root = ../000-main.tex
\chapter{Introduction to interpretability in machine learning}

\section{Model interpretability and variable relevance}

In a stimulating and provocative paper, Breiman (2001)
% TODO: cite
shook the
statistical community by making it to be aware that traditional
Statistics was no longer the only way to learn from data:
\begin{itemize}
	\item \emph{Data modelling culture} (traditional statistics):
	      \begin{itemize}
		      \item Linear regression, logistic regression, additive models, etc.
		      \item They allow to interpret how the response variable is associated with
		            the input variables: \iemph{Transparent models}.
	      \end{itemize}
	\item \emph{Algorithmic Modeling Culture} (machine learning):
	      \begin{itemize}
		      \item Decision trees, neural networks, support vector machines, etc.
		      \item They have extremely good predictive accuracy, and they usually outperform
		            in this criterion statistical models.
		      \item However, they have low interpretability: \iemph{Black-box models}.
	      \end{itemize}
\end{itemize}

From that it follows an apparent dichotomy: \iemph{predictive capacity} versus \iemph{interpretability}.

Breiman claimed for procedures allowing better interpretation of the algorithmic models
results, without giving up their predictive ability.

Machine learning community has been worried about interpretability:
\begin{quote}
	If the users do not trust a model or a prediction, they will not use it.

	\emph{Ribeiro, Singh and Guestrin (2016)}
\end{quote}

In 2018, the General Data Protection Regulation (GDPR) of the European Union
established users' right to explanation: ``When an algorithmic decision
significantly affects a user, he or she has the right to ask for an explanation
of such decision''.

A powerful research line has been developed:
\iemph{Interpretable Machine Learning} (\iemph{IML}) and
\iemph{eXplainable Artificial Intelligence} (\iemph{XAI}).

There are several review papers on this topic.
Barredo-Arrieta et al. (2020) is one of the most
recent and extensive ones. % TODO: cite properly
Also three monographs: Molnar (2019), Biecek and Burzykowski (2021), and
Masís (2021).

There are also many functions and packages in both R and Python.

\section{IML/XAI concepts}

\begin{note}
	The \emph{accuracy} of predictions is no longer the only criterion to evaluate
	the quality of a prediction algorithm.
\end{note}

Desirable properties for predictive models:
\begin{itemize}
	\item \iemph{transparency}
	\item \iemph{interpretability}
	\item \iemph{explainability}
\end{itemize}

However, these concepts are not well defined and are difficult to measure:
\begin{itemize}
	\item Lipton 2018: ``The term \emph{interpretability} is ill-defined''
	\item Barredo-Arrieta et al. 2020: ``The derivation of general metrics
	      to assess the quality of XAI approaches remain as an open challenge.''
\end{itemize}

To fix ideas, the possibility of obtaining information on the
performance of the algorithm, in both the \emph{global} and \emph{local} senses, is
now appreciated.

\subsection{Global vs. Local interpretability}

\begin{definition}{Global interpretability}{}
	Measures of variable importance or relevance.

	Information about the global performance refers to determining which
	is the role of each explanatory variable in the prediction
	process over the whole support of the explanatory variables.
\end{definition}

\begin{definition}{Local interpretability}{}
	Why the prediction model does a particular prediction for a given individual?

	The aim is to provide a meaningful explanation of why the algorithm returns a
	certain prediction given a particular combination of the predicting variable values.

	\tcblower

	\begin{note}
		The local aspect of interpretability is directly related with the users’
		right to explanation advocated for by, for instance, the EU’s GDPR.
	\end{note}
\end{definition}

\subsection{Transparent models versus ``black box'' models}

\section{Global methods (relevance of variables) versus local methods (explainability)}
